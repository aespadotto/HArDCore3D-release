H\+Ar\+D\+::\+Core3D (Hybrid Arbitrary Degree\+::\+Core 3D) -\/ Library to implement schemes with face and cell polynomial unknowns on 3D generic meshes.

This is the 3D version of the H\+Ar\+D\+::\+Core library (\href{https://github.com/jdroniou/HArDCore}{\texttt{ https\+://github.\+com/jdroniou/\+H\+Ar\+D\+Core}}). The documentation can be found at \href{https://jdroniou.github.io/HArDCore3D-release/}{\texttt{ https\+://jdroniou.\+github.\+io/\+H\+Ar\+D\+Core3\+D-\/release/}}

The meshing (src/\+Mesh/\+Stem\+Mesh) and quadrature (src/\+Quadratures) modules in \mbox{\hyperlink{namespaceHArDCore3D}{H\+Ar\+D\+Core3D}} are partially based on Marco Manzini\textquotesingle{}s code available at \href{https://github.com/gmanzini-LANL/PDE-Mesh-Manager}{\texttt{ https\+://github.\+com/gmanzini-\/\+L\+A\+N\+L/\+P\+D\+E-\/\+Mesh-\/\+Manager}}.

The purpose of H\+Ar\+D\+::\+Core3D is to provide easy-\/to-\/use tools to code hybrid schemes, such as the Hybrid High-\/\+Order method. The data structure is described using intuitive classes that expose natural functions we use in the mathematical description of the scheme. For example, each mesh element is a member of the class \textquotesingle{}Cell\textquotesingle{}, that gives access to its diameter, the list of its faces (themselves members of the class \textquotesingle{}Face\textquotesingle{} that describe the geometrical features of the face), etc. Functions are also provided to compute the key elements usually required to implement hybrid schemes, such as mass matrices of local basis functions, stiffness matrices, etc. The approach adopted is that of a compromise between readibility/usability and efficiency.

As an example, when creating a mass matrix, the library requires the user to first compute the quadrature nodes and weights, then compute the basis functions at these nodes, and then assemble the mass matrix. This ensures a local control on the required degree of exactness of the quadrature rule, and also that basis functions are not evaluated several times at the same nodes (once computed and stored locally, the values at the quadrature nodes can be re-\/used several times). Each of these steps is however concentrated in one line of code, so the assembly of the mass matrix described above is actually done in three lines\+:


\begin{DoxyCode}{0}
\DoxyCodeLine{QuadratureRule quadT = generate\_quadrature\_rule(T, 2*m\_K);<br>}
\DoxyCodeLine{boost::multi\_array<double, 2> phiT\_quadT = evaluate\_quad<Function>::compute(basisT, quadT);<br>}
\DoxyCodeLine{Eigen::MatrixXd MTT = compute\_gram\_matrix(phiT\_quadT, quadT);}
\end{DoxyCode}


Note that the {\ttfamily Element\+Quad} class offers a convenient way to compute and store the quadrature rules and values of basis functions at the nodes, and makes it easy to pass these data to functions. More details and examples are provided in the documentation.

The implementations in this library follow general principles described in the appendix of the book \char`\"{}\+The Hybrid High-\/\+Order Method for Polytopal Meshes\+: Design, Analysis, and Applications\char`\"{} (D. A. Di Pietro and J. Droniou. Number 19 in Modeling, Simulation and Applications, Springer International Publishing, 2020. I\+S\+BN 978-\/3-\/030-\/37202-\/6 (Hardcover) 978-\/3-\/030-\/37203-\/3 (e\+Book). url\+: \href{https://hal.archives-ouvertes.fr/hal-02151813}{\texttt{ https\+://hal.\+archives-\/ouvertes.\+fr/hal-\/02151813}}). High-\/order methods with hybrid unknowns have certain specificities which sometimes require fine choices, e.\+g. of basis functions (hierarchical, orthonormalised or not), etc. We refer to the aformentioned manuscript for discussion on these specificities. If using the code provided here, or part thereof, for a scientific publication, please refer to this book for details on the implementation choices.

This library was developed with the direct help and indirect advice of several people. Many thanks to them\+: Daniel Anderson, Hanz Martin Cheng, Daniele Di Pietro, Lorenzo Botti.

The development of this library was partially supported by Australian Government through the Australian Research Council\textquotesingle{}s Discovery Projects funding scheme (project number D\+P170100605). 